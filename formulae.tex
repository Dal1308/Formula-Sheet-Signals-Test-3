\documentclass{article}

% Packages for mathematical typesetting
\usepackage{amsmath}
\usepackage{amsfonts}
\usepackage{amssymb}

\begin{document}

% Title and author
\title{Test Document}
\author{Your Name}
\date{\today}
\maketitle

% Introduction or instructions for the test
\section{Instructions}
Insert instructions or introduction here.

% Questions with formulas
\section{Questions}
\begin{enumerate}
    \item \textbf{Question 1:} State the Pythagorean theorem and prove it.
    
    \item \textbf{Question 2:} Find the derivative of $f(x) = \sin(x^2)$ with respect to $x$.
    
    \item \textbf{Question 3:} Solve the following equation:
    \[
    x^2 - 5x + 6 = 0
    \]
    
    \item \textbf{Question 4:} Evaluate the limit:
    \[
    \lim_{x \to 0} \frac{\sin(x)}{x}
    \]
\end{enumerate}

% Solutions (if needed)
\section{Solutions}
\begin{enumerate}
    \item \textbf{Question 1:} The Pythagorean theorem states that in a right triangle, the square of the length of the hypotenuse ($c$) is equal to the sum of the squares of the lengths of the other two sides ($a$ and $b$). Mathematically, it can be expressed as:
    \[
    c^2 = a^2 + b^2
    \]
    To prove it, one can use similar triangles or the area of squares constructed on the sides of the triangle.
    
    \item \textbf{Question 2:} The derivative of $f(x) = \sin(x^2)$ with respect to $x$ can be found using the chain rule:
    \[
    \frac{d}{dx}\left(\sin(x^2)\right) = \cos(x^2) \cdot 2x = 2x\cos(x^2)
    \]
    
    \item \textbf{Question 3:} The equation $x^2 - 5x + 6 = 0$ can be factored as $(x - 2)(x - 3) = 0$, so the solutions are $x = 2$ and $x = 3$.
    
    \item \textbf{Question 4:} Using L'Hôpital's rule or trigonometric limits, we find:
    \[
    \lim_{x \to 0} \frac{\sin(x)}{x} = 1
    \]
\end{enumerate}

\end{document}
